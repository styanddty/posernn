\documentclass[12pt]{article}

\usepackage{tikz}

%used for drawing flow chart
\usepackage{tikz}
\usetikzlibrary{shapes,arrows,positioning,fit,calc}


\begin{document}

\title{Pose Estimation as Sequences \\ \small CS231A Project Proposal}
\author{Matthew Chen\\
  \small \vspace{-2mm} Department of Computer Science\\
  \small \vspace{-2mm} Stanford University\\
  \small mcc17@stanford.edu
  \and
  Melvin Low\\
  \small \vspace{-2mm} Department of Computer Science\\
  \small \vspace{-2mm} Stanford University\\
  \small mwlow@stanford.edu}

\date{}
\maketitle

\section{Problem}

For this project we will work on models for estimating single human pose in RGB images. We will be using the MPII human pose estimation dataset which is composed of around 25,000 images annotated with human poses. The poses take the form of pixels coordinates which correspond to a fixed number of joints for a given person in an image. Our goal would be to train a model which takes as input a given image with a single person and outputs their corresponding joint coordinates.

Pose estimation has been a long studied problem in computer vision as it provides information which is essential for interaction in a social environment. For instance understanding of pose can be a important input in activity recognition and motion planning, leading to more accurate models of human activity. In the past the state of the art models involved variants of the deformable parts model. In these models fast part detectors were convolved across the image and a candidate output was chosen which presented a coherent joint configuration \cite{felzenszwalb2010object}.

More recently, building on the success of convolutional neural networks in computer vision, several papers have used CNN techniques to achieve new state of the art results. DeepPose framed the problem as a regression problem where joint coordinates were output from a CNN as a offset from an achor point and a casade of networks to refine the predictions \cite{toshev2014deeppose}. Another method in \cite{carreira2015human} iteratively refined an initial prior pose to match the pose in the picture by iterating on error offsets for joints until convergence. Deepcut presented the problem as optimizing a global objective where CNNs were used to define unary probabilities for nodes in a PGM \cite{pishchulin2015deepcut}. This approach is unique in that it allows for pose estimation of multiple people in the image. More recent methods aim to include both local and global features into a single network with differentiable layers to optimize a joint objective \cite{wei2016convolutional} \cite{newell2016stacked}. For this project we will model the dependencies between joints in the pose with an Long-Short Term Memory (LSTM) model.

\section{Technical Plan}

Many of the convolutional neural networks stated previously make a independence assumption on joint location given the image features during the first inference pass. The latter methods, which involve iterative refinement, can be thought of as encoding dependencies between joint locations through successive passes through the network. In our approach we aim to model dependencies between joints explicitly by modeling the pose as a sequence of coordinates. The inspiration for this approach is the relative success of image captioning models. In such models a CNN is used to extract global image features and generates a caption based on word embeddings. The dependencies between words is essential in this model to generate coherent sentence structures.

Our approach builds on this idea by modelling the pose parameters as a coherent full body pose. In our LSTM network we will set a fixed sequence of joint locations as the output and extract these locations based on image features and the hidden state encoded by the LSTM. A illustration of our proposed model is outlined in Figure \ref{model_diag}.

The work will be divided evenly amongst the two team members for all components of this project. We will be using TensorFlow as our main framework to build the deep learning architecture. Additionally we will use standard python libraries such as numpy for matrix representations and image libraries for basic preprocessing. The novel component will be modelling the dependency structure of the joint locations with an LSTM model.

%define flowchart elements
\tikzstyle{pipe} = [rectangle, draw, fill=orange!30, rounded corners, text width=4.5em, text height=2em, text badly centered, node distance=7em, inner sep=0pt]
\tikzstyle{line} = [draw, -latex']


\begin{figure}
\begin{tikzpicture}
\node[pipe] (img) {Image};
\node[pipe, right of=img] (CNN) {CNN};
\node[pipe, right of=CNN] (lstm1) {LSTM};
\node[pipe, right of=lstm1] (lstm2) {LSTM};
\node[pipe, right of=lstm2] (lstm3) {LSTM};

\node[pipe, above of=lstm1] (head) {Head};
\node[pipe, above of=lstm2] (sh1) {Shoulder1};
\node[pipe, above of=lstm3] (sh2) {Shoulder2};


\path[line] (img)--(CNN);
\path[line] (CNN)--(lstm1);
\path[line] (lstm1)--(lstm2);
\path[line] (lstm2)--(lstm3);

\path[line] (lstm1)--(head);
\path[line] (lstm2)--(sh1);
\path[line] (lstm3)--(sh2);

\end{tikzpicture}
\label{model_diag}
\caption{Model diagram. A CNN is used to extract featuers from the image which is then fed into an LSTM which is able to output joint coordinates in pixel space.}
\end{figure}

\section{Milestones}

Our approach to this problem would be to first explore the MPII dataset and establish a baseline model. In this case we choose DeepPose as our baseline as it is one of the simplest CNN based pose estimation models. Then we will build our model and evaluate the results compared to our baseline.

\begin{enumerate}
\item Download and create processing scripts for MPII data
\item Run baseline DeepPose regression model on task
\item Incorporate LSTM to model dependencies between joints
\item Evaluate results and look at failure modes
\end{enumerate}

\nocite{*}
\bibliographystyle{ieee}
\bibliography{proposal}

\end{document}